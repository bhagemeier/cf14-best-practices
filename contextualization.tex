\label{part_contextualization}
\part{Contextualization}

%%%%%%%%%%%%%%%%%%%%%%%%%%%%%%%%%%%%%%%%%%%%%%%%%%%%%%%%%%%%%%%%%%%%%%%%%%%
\begin{frame}
  \frametitle{Contextualization}
  \framesubtitle{What is it?}

  A way to
  \begin{itemize}
    \item inject user credentials on boot
    \item perform simple configuration tasks
    \item customize generic images on boot
    \item pass arbitrary data into virtual machine instances
  \end{itemize}

  \hfill\\

  \textbf{NOT} a way to
  \begin{itemize}
    \item transfer large amounts of data
    \item perform complex or interactive tasks
    \item replace orchestration or provisioning tools
  \end{itemize}
\end{frame}

%%%%%%%%%%%%%%%%%%%%%%%%%%%%%%%%%%%%%%%%%%%%%%%%%%%%%%%%%%%%%%%%%%%%%%%%%%%
\begin{frame}
  \frametitle{Contextualization}
  \framesubtitle{How it works?}

  \begin{enumerate}
    \item Data is passed to the instance at boot
    \item The instance is expected to use it
    \item Implementation is usually platform-specific
    \item There are attempts to "standardize" the process
  \end{enumerate}

  \hfill\\

  \textbf{Delivery mechanisms}: custom-made scripts, direct injection, metadata services, contextualization disks
\end{frame}

%%%%%%%%%%%%%%%%%%%%%%%%%%%%%%%%%%%%%%%%%%%%%%%%%%%%%%%%%%%%%%%%%%%%%%%%%%%
\begin{frame}
  \frametitle{Contextualization}
  \framesubtitle{Using \textit{cloud-init}}

  \begin{itemize}
    \item \textit{De facto} standard for multi-platform contextualization
    \item Supports multiple delivery mechanisms $\rightarrow$ "datasources"
    \item Offers implementations of commonly performed tasks $\rightarrow$ "modules"
    \item Configuration in the form of a YAML file $\rightarrow$ "user\_data"
  \end{itemize}

  \hfill\\

  For details, see \href{http://cloudinit.readthedocs.org/en/latest/index.html}{\nolinkurl{http://cloudinit.readthedocs.org/en/latest/index.html}}
\end{frame}

%%%%%%%%%%%%%%%%%%%%%%%%%%%%%%%%%%%%%%%%%%%%%%%%%%%%%%%%%%%%%%%%%%%%%%%%%%%
\begin{frame}
  \frametitle{Contextualization}
  \framesubtitle{Using provisioning}

  \begin{itemize}
    \item Advanced task automation and parallelization
    \item Typically using a declarative approach
    \item Slow start, considerable advantages later on
    \item Preferred method for the cloud environment
  \end{itemize}

  \hfill \\

  \textbf{Tools}: puppet, chef, ansible, salt, CFEngine, \dots
\end{frame}

%%%%%%%%%%%%%%%%%%%%%%%%%%%%%%%%%%%%%%%%%%%%%%%%%%%%%%%%%%%%%%%%%%%%%%%%%%%
\begin{frame}[fragile]
  \frametitle{Contextualization}
  \framesubtitle{Usage example -- Simple}

  \begin{lstlisting}
#cloud-config

groups:
  - cloud-users

users:
  - name: cloudman
    gecos: Cloud H. Man
    sudo: ALL=(ALL) NOPASSWD:ALL
    groups: cloud-users, admin
    lock-passwd: true
    ssh-authorized-keys:
      - <ssh pub key 1>
      - <ssh pub key 2>
  \end{lstlisting}
\end{frame}

%%%%%%%%%%%%%%%%%%%%%%%%%%%%%%%%%%%%%%%%%%%%%%%%%%%%%%%%%%%%%%%%%%%%%%%%%%%
\begin{frame}[fragile]
  \frametitle{Contextualization}
  \framesubtitle{Usage example -- Advanced}

  \begin{lstlisting}
#cloud-config

mounts:
- [vdc,/data,ext4]

package_upgrade: true

packages:
- qemu-guest-agent

phone_home:
  url: http://my.master.org/instances/$INSTANCE_ID
  post: [ pub_key_rsa, instance_id ]

final_message: "The system is ready!"
  \end{lstlisting}
\end{frame}

%%%%%%%%%%%%%%%%%%%%%%%%%%%%%%%%%%%%%%%%%%%%%%%%%%%%%%%%%%%%%%%%%%%%%%%%%%%
\begin{frame}[fragile]
  \frametitle{Contextualization}
  \framesubtitle{Usage example -- Provisioning}

  \begin{lstlisting}
#cloud-config

puppet:
 conf:
   agent:
     server: "my.master.org"
     certname: "%i.%f"

   ca_cert: |
     -----BEGIN CERTIFICATE-----
     ... omitted ...
     -----END CERTIFICATE-----
  \end{lstlisting}
\end{frame}

%%% Local Variables:
%%% TeX-master: "2014-05-23_Best_Practices"
%%% End:
