\part{Images}

%%%%%%%%%%%%%%%%%%%%%%%%%%%%%%%%%%%%%%%%%%%%%%%%%%%%%%%%%%%%%%%%%%%%%%%%%%%
\begin{frame}
\frametitle{What are images?}
\framesubtitle{}
There are various answers to this question
\begin{itemize}
\item Typically
  \begin{itemize}
  \item A blob of data representing the contents of a disk or file system
  \item Usually contains the root file system contents of an operating system
  \end{itemize}
\item Data
\item Image is usually used as a \emph{template} to create a virtual
  machine or \emph{instance} of it.
%% another wonderful diagram showing the relations among image and instance
%% will add additional relations later on, e.g. block storage volumes
\end{itemize}
\end{frame}

%%%%%%%%%%%%%%%%%%%%%%%%%%%%%%%%%%%%%%%%%%%%%%%%%%%%%%%%%%%%%%%%%%%%%%%%%%%
\begin{frame}
\frametitle{Image Creation}
\framesubtitle{Basic Steps}
Starting points
\begin{itemize}
\item Existing images
\item Installation medium
\item Special installers
\end{itemize}
Our recommendation
\begin{itemize}
\item Existing images
  \begin{itemize}
  \item Basic images of popular LINUX distributions available
    \begin{itemize}
    \item Ubuntu
    \item Debian
    \item ScientificLinux
    \end{itemize}
  \end{itemize}
\end{itemize}
\end{frame}

%%%%%%%%%%%%%%%%%%%%%%%%%%%%%%%%%%%%%%%%%%%%%%%%%%%%%%%%%%%%%%%%%%%%%%%%%%%
\begin{frame}
\frametitle{Image Creation}
\framesubtitle{Starting from existing images}


\end{frame}

%%%%%%%%%%%%%%%%%%%%%%%%%%%%%%%%%%%%%%%%%%%%%%%%%%%%%%%%%%%%%%%%%%%%%%%%%%%
\begin{frame}[fragile]
\frametitle{Image Creation}
\framesubtitle{Starting from scratch}
An example
\begin{lstlisting}
  $ truncate -s 1G debian-7.4.0.img
  $ kvm -cdrom debian-7.4.0-amd64-netinst.iso debian-7.4.0.img
\end{lstlisting}
During installation
\begin{itemize}
\item Select ``Install''
\item Give a strong root password despite it being locked later on
\item Use manual partitioning
  \begin{itemize}
  \item Avoid creating a swap partition and put everything in one partition
  \item Additional block devices can and should be added at runtime
  \end{itemize}
\item Software installation
  \begin{itemize}
  \item Install as few packages as possible
  \end{itemize}
\item Boot loader on MBR
\end{itemize}
\end{frame}

%%%%%%%%%%%%%%%%%%%%%%%%%%%%%%%%%%%%%%%%%%%%%%%%%%%%%%%%%%%%%%%%%%%%%%%%%%%
\begin{frame}
\frametitle{Image Creation}
\framesubtitle{Final Cleanup}
\begin{itemize}
\item Passwords
\item Log files
\item Excessive data
  \begin{itemize}
  \item Find it using du or ncdu
  \end{itemize}
\item ...?
\end{itemize}
We posted the general procedure in the EGI Blog: \url{http://goo.gl/ju7vgP}
\end{frame}

%%%%%%%%%%%%%%%%%%%%%%%%%%%%%%%%%%%%%%%%%%%%%%%%%%%%%%%%%%%%%%%%%%%%%%%%%%%
\begin{frame}
\frametitle{Image Creation}
\framesubtitle{Packaging}
\begin{itemize}
\item zerofree
\item create compressed qcow2
\end{itemize}
More details in the EGI Blog: \url{http://goo.gl/UA6t1Y}
\end{frame}


%%%%%%%%%%%%%%%%%%%%%%%%%%%%%%%%%%%%%%%%%%%%%%%%%%%%%%%%%%%%%%%%%%%%%%%%%%%
\begin{frame}
\frametitle{Image Creation}
\framesubtitle{Updates}
\begin{itemize}
\item Steps to update contents of an image and create a new one
\item Is this really possible with \emph{only OCCI} commands?
  \begin{itemize}
  \item We'd certainly want to support this, i.\,e. take an instance,
    create a snapshot of it and use thas as a new image.
  \end{itemize}
\item It should be just as easy as creating an image from an existing one.
\end{itemize}
\end{frame}
